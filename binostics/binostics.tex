% !TeX root = RJwrapper.tex
\title{binostics: computing scagnostics measures in R and C++}
\author{by Ursula Laa, Dianne Cook, Hadley Wickham, Heike Hofmann}

\maketitle

\abstract{%
An abstract of less than 150 words.
}

\hypertarget{introduction}{%
\subsection{Introduction}\label{introduction}}

\begin{itemize}
\tightlist
\item
  scatter plot matrix, need for variable selection with big data
\item
  scagnostics \citep[\citet{WW08}]{scag}: characterise scatterplots with
  8 summaries (scatterplot diagnostics), select interesting plots from a
  SPLOM of the scagnostics measures and interactively look at those
\item
  similar methods (e.g.~MIC\citep{Reshef1518}, available via minerva
  package \citep{minerva})
\item
  alternative methods to see combinations of variables: tours
  \citep[\citet{tourr}]{As85}, PCA, LDA, PP \citep{f87}
\end{itemize}

things to mention: scagnostics package (CRAN) \citep{LWscagR}, mbgraphic
\citep{mbgraphic} (archived!) and Katrins thesis work \citep{Grimm2016}

\hypertarget{scagnostics-measures}{%
\subsection{Scagnostics measures}\label{scagnostics-measures}}

Following \citet{scag} the scagnostics measures are evaluated on the
data after hexagon binning, and based on the Delaunay triangulation for
comutational efficiency. In addition, outlying points are removed before
computing the measures (with exception of the Outlying measure), to make
the measures more robust.

\hypertarget{underlying-definitions}{%
\subsubsection{Underlying definitions}\label{underlying-definitions}}

All measures are based on graphs, i.e.~a set of vertices and edges, that
can all be extracted from the Delaunay triangulation, namely:

\begin{itemize}
\tightlist
\item
  the convex hull, i.e.~the outer edges of the Delaunay triangulation
\item
  the alpha hull \citep{1056714}
\item
  the minimum spanning tree (MST) \citep{kruskal1956}
\end{itemize}

In the following, the length of the MST is defined as the sum of lengths
of all edges, and \(q_x\) is defined as the \(x\)th percentile of the
MST edge lengths.

When computing the alpha hull, \(\alpha\) is set to \(q_{90}\)
\citep{scag}. The definition of outlying points is also based on edge
length, with points being tagged as outlying if all adjacent edges in
the MST have a lenght larger than \begin{equation}
w = q_{75} + 1.5 ( q_{75} - q_{25}).
\end{equation} Several of the measures are corrected for dependence on
sample size using the weight \begin{equation}
\omega = 0.7 + \frac{0.3}{1+t^2},
\end{equation} where \(t = n / 500\) and \(n\) the sample size. This
correction weight was determined by comparing the scagnostics measures
over a large number of datasets \citep{scag}.

\hypertarget{measure-definitions}{%
\subsubsection{Measure definitions}\label{measure-definitions}}

\begin{itemize}
\tightlist
\item
  \textbf{Outlying}: compares the edge lengths of the MST of outlying
  points with the length of the original MST T \begin{equation}
  c_{\mathrm{outlying}} = \frac{length(T_{\mathrm{outliers}})}{length(T)}
  \end{equation}
\item
  \textbf{Skewed}: is measuring skewness in the distribution of MST edge
  lengths (and thus large values might not correspond to skewed
  distributions of points) as \begin{equation}
  c'_{\mathrm{skewed}} = \frac{q_{90}-q_{50}}{q_{90}-q_{10}},
  \end{equation} and is corrected to adjust for dependence on the sample
  size as \begin{equation}
  c_{\mathrm{skewed}} = 1 - \omega (1 - c'_{\mathrm{skewed}}).
  \end{equation}
\item
  \textbf{Sparse}: detects if points are only found in small number of
  locations in the plane, as is the case for categorical variables,
  \begin{equation}
  c_{\mathrm{sparse}} = \omega q_{90}.
  \end{equation}
\item
  \textbf{Clumpy}: to detect clustering we split the MST in two parts by
  removing a single edge \(j\), and compare the largest edge lenght
  within the smaller of the two subsets to the length of the removed
  edge \(j\). The clumpy measure is defined by maximising over all edges
  in the MST as \begin{equation}
  c_{\mathrm{clumpy}} = \max_j [1 - \max_k[\mathrm{length}(e_k)] / \mathrm{length}(e_j) ],
  \end{equation} with \(k\) running over all edges in the smaller
  subgraph found after removing edge \(j\) from the MST.
\item
  \textbf{Striated}: aims to detect patterns like smooth algebraic
  functions or parallel lines, by evaluating the angles between adjacent
  edges, \begin{equation}
  c_{\mathrm{striated}} = \frac{1}{|V|} \sum_{v\in V^{(2)}} I(\cos\theta_{e(v,a)e(v,b)} < -0.75),
  \end{equation} where \(V\) is the set of vertices, \(|V|\) the total
  number of vertices in the triangulation, \(V^{(2)}\) the subset of
  vertices with two edges (i.e.~vertices of degree two), and \(I\) the
  indicator function.
\item
  \textbf{Convex}: convexity is computed as the ratio of the area of the
  alpha hull \(A\) and the convex hull \(H\), adjusting for sample size
  dependence, \begin{equation}
  c_{\mathrm{convex}} = \omega \frac{area(A)}{area(H)}.
  \end{equation}
\item
  \textbf{Skinny}: The ratio of the perimeter to the area of the alpha
  hull \(A\), with normalization such that zero corresponds to a full
  circle and values close to 1 indicate a skinny polygon,
  \begin{equation}
  c_{\mathrm{skinny}} = 1 -  \frac{\sqrt{4\pi area(A)}}{perimeter(A)}.
  \end{equation}
\item
  \textbf{Stringy}: a stringy distribution should have no branches in
  the MST. This is evaluated by counting the number of vertices of
  degree two and comparing them to the total number of vertices
  (dropping those of degree one), \begin{equation}
  c_{\mathrm{stringy}} = \frac{|V^{(2)}|}{|V| - |V^{(1)}|}.
  \end{equation}
\item
  \textbf{Monotonic}: monotonicity is evaluated via the squared Spearman
  correlation coefficient, i.e.~the Pearson correlation between the
  ranks of \(x\) and \(y\), \begin{equation}
  c_{\mathrm{monotonic}} = r_{\mathrm{Spearman}}^2.
  \end{equation}
\end{itemize}

XXX maybe there is a problem with the implementation of outlying,
because this sometimes gives values larger than one? (e.g.~vs vs am on
mtcars, also gives NaN for several other measures: clumpy, striated,
stringy, monotonic)

\hypertarget{implementation}{%
\subsection{Implementation}\label{implementation}}

\begin{itemize}
\tightlist
\item
  input/output structure, how are we combining variables
\item
  c++ interface for efficient binning, triangulation
\item
  computation of measures
\item
  how to work with the resulting output
\end{itemize}

Input is either: as raw function takes two vectors to calculate the
scagnositcs for, or a 2D array (matrix or data frame) for which all
combinations of variables will be considered

Output: the raw function (called on two vecotrs) returns a list which
contains the scagnostics measures (\texttt{s}) and the binning
information (\texttt{bins}), when called on a 2D array the output is of
the ``scagdf'' class. The main result is stored in matrix format, with
each column corresponding to one of the scagnositcs measures, and each
row a combination of input variables. The row names specify this
combination, e.g.~\texttt{scagdf{[}"x\ vs\ y",{]}} returns all
scagnostics measures computed on the scatterplot of variable \texttt{x}
vs variable \texttt{y}. Additional attributes are ``vars'' (specifying
the assignment of combinations of input variables to rows in the output)
and ``data'' (a copy of the input data).

The R interface handels the reading the input and output formatting, and
for each combination of variables we call C++ functionalities to compute
the measures. The steps are:

\begin{itemize}
\tightlist
\item
  hexbinning of the data (here number of bins is free parameter, but if
  we find more than 250 non-empty bins in the result, redo binning with
  half the number of bins on each axis)
\item
  computing the Delaunay triangulation and the MST
\item
  use cutoff on edge length (based on MST) to identify outlying points,
  and recompute DT after removing outliers
\item
  compute scagnostics measures
\end{itemize}

\hypertarget{example}{%
\subsection{Example}\label{example}}

A simple example showcasing how to use scagnostics measures

\begin{Schunk}
\begin{Sinput}
# FIXME need better examples!

library(tidyverse)
s <- binostics::scagnostics(mtcars)
s_tibble <- tibble::as_tibble(s[1:nrow(s),]) %>% #get tibble for plotting
  dplyr::mutate(vars = rownames(s))
GGally::ggpairs(select(s_tibble, -vars))
filter(s_tibble, Skinny==1 & Convex==1)$vars # these are discrete values only on the "outside"
filter(s_tibble, Skinny==1 & Convex==0)$vars # other discrete values give Convex=0
filter(s_tibble, Outlying==2 & Skewed==0)$vars # also points to discrete values only on the "outside"
\end{Sinput}
\end{Schunk}

\hypertarget{summary}{%
\subsection{Summary}\label{summary}}

\begin{itemize}
\tightlist
\item
  scagnostics measures useful when exploring large datasets
\item
  the binostics implementaton is efficient thanks to c++ interface, and
  portable (no java dependence)
\item
  most useful for interactive exploration (maybe good to use in Shiny
  app?)
\item
  connection with PP (cite PPI paper)
\end{itemize}

\bibliography{RJreferences}


\address{%
Ursula Laa\\
Affiliation\\
line 1\\ line 2\\
}
\href{mailto:author1@work}{\nolinkurl{author1@work}}

\address{%
Dianne Cook\\
Affiliation\\
line 1\\ line 2\\
}
\href{mailto:author2@work}{\nolinkurl{author2@work}}

\address{%
Hadley Wickham\\
Affiliation\\
line 1\\ line 2\\
}
\href{mailto:author3@work}{\nolinkurl{author3@work}}

\address{%
Heike Hofmann\\
Affiliation\\
line 1\\ line 2\\
}
\href{mailto:author4@work}{\nolinkurl{author4@work}}

