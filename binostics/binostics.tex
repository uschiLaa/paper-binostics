% !TeX root = RJwrapper.tex
\title{binostics: computing scagnostics measures in R and C++}
\author{by Ursula Laa, Dianne Cook, Hadley Wickham, Heike Hofmann}

\maketitle

\abstract{%
An abstract of less than 150 words.
}

\hypertarget{introduction}{%
\subsection{Introduction}\label{introduction}}

\begin{itemize}
\tightlist
\item
  scatter plot matrix, need for variable selection with big data
\item
  scagnostics \citep[\citet{WW08}]{scag}: characterise scatterplots with
  8 summaries (scatterplot diagnostics), select interesting plots from a
  SPLOM of the scagnostics measures and interactively look at those
\item
  similar methods (e.g.~MIC\citep{Reshef1518}, available via minerva
  package \citep{minerva})
\item
  alternative methods to see combinations of variables: tours
  \citep[\citet{tourr}]{As85}, PCA, LDA, PP \citep{f87}
\end{itemize}

things to mention: scagnostics package (CRAN) \citep{LWscagR}, mbgraphic
\citep{mbgraphic} (archived!) and Katrins thesis work \citep{Grimm2016}

\hypertarget{scagnostics-measures}{%
\subsection{Scagnostics measures}\label{scagnostics-measures}}

Following \citet{scag} the scagnostics measures are evaluated on the
data after hexagon binning, and based on the Delaunay triangulation for
comutational efficiency. In addition, outlying points are removed before
computing the measures (with exception of the Outlying measure), to make
the measures more robust.

\hypertarget{underlying-definitions}{%
\subsubsection{Underlying definitions}\label{underlying-definitions}}

All measures are based on graphs, i.e.~a set of vertices and edges, that
can all be extracted from the Delaunay triangulation, namely:

\begin{itemize}
\tightlist
\item
  the convex hull, i.e.~the outer edges of the Delaunay triangulation
\item
  the alpha hull \citep{1056714}
\item
  the minimum spanning tree (MST) \citep{kruskal1956}
\end{itemize}

In the following, the length of the MST is defined as the sum of lengths
of all edges, and \(q_x\) is defined as the \(x\)th percentile of the
MST edge lengths.

When computing the alpha hull, \(\alpha\) is set to \(q_{90}\)
\citep{scag}. The definition of outlying points is also based on edge
length, with points being tagged as outlying if all adjacent edges in
the MST have a lenght larger than \begin{equation}
w = q_{75} + 1.5 ( q_{75} - q_{25}).
\label{eq:w}
\end{equation} Several of the measures are corrected for dependence on
sample size using the weight \begin{equation}
\omega = 0.7 + \frac{0.3}{1+t^2},
\end{equation} where \(t = n / 500\) and \(n\) the sample size. This
correction weight was determined by comparing the scagnostics measures
over a large number of datasets \citep{scag}.

\hypertarget{measure-definitions}{%
\subsubsection{Measure definitions}\label{measure-definitions}}

\begin{itemize}
\tightlist
\item
  \textbf{Outlying}: compares the edge lengths of the MST of outlying
  points with the length of the original MST T \begin{equation}
  c_{\mathrm{outlying}} = \frac{length(T_{\mathrm{outliers}})}{length(T)}
  \end{equation}
\item
  \textbf{Skewed}: is measuring skewness in the distribution of MST edge
  lengths (and thus large values might not correspond to skewed
  distributions of points) as \begin{equation}
  c'_{\mathrm{skewed}} = \frac{q_{90}-q_{50}}{q_{90}-q_{10}},
  \end{equation} and is corrected to adjust for dependence on the sample
  size as \begin{equation}
  c_{\mathrm{skewed}} = 1 - \omega (1 - c'_{\mathrm{skewed}}).
  \end{equation}
\item
  \textbf{Sparse}: detects if points are only found in small number of
  locations in the plane, as is the case for categorical variables,
  \begin{equation}
  c_{\mathrm{sparse}} = \omega q_{90}.
  \end{equation}
\item
  \textbf{Clumpy}: to detect clustering we split the MST in two parts by
  removing a single edge \(j\), and compare the largest edge lenght
  within the smaller of the two subsets to the length of the removed
  edge \(j\). The clumpy measure is defined by maximising over all edges
  in the MST as \begin{equation}
  c_{\mathrm{clumpy}} = \max_j [1 - \max_k[\mathrm{length}(e_k)] / \mathrm{length}(e_j) ],
  \end{equation} with \(k\) running over all edges in the smaller
  subgraph found after removing edge \(j\) from the MST.
\item
  \textbf{Striated}: aims to detect patterns like smooth algebraic
  functions or parallel lines, by evaluating the angles between adjacent
  edges, \begin{equation}
  c_{\mathrm{striated}} = \frac{1}{|V|} \sum_{v\in V^{(2)}} I(\cos\theta_{e(v,a)e(v,b)} < -0.75),
  \end{equation} where \(V\) is the set of vertices, \(|V|\) the total
  number of vertices in the triangulation, \(V^{(2)}\) the subset of
  vertices with two edges (i.e.~vertices of degree two), and \(I\) the
  indicator function.
\item
  \textbf{Convex}: convexity is computed as the ratio of the area of the
  alpha hull \(A\) and the convex hull \(H\), adjusting for sample size
  dependence, \begin{equation}
  c_{\mathrm{convex}} = \omega \frac{area(A)}{area(H)}.
  \end{equation}
\item
  \textbf{Skinny}: The ratio of the perimeter to the area of the alpha
  hull \(A\), with normalization such that zero corresponds to a full
  circle and values close to 1 indicate a skinny polygon,
  \begin{equation}
  c_{\mathrm{skinny}} = 1 -  \frac{\sqrt{4\pi area(A)}}{perimeter(A)}.
  \end{equation}
\item
  \textbf{Stringy}: a stringy distribution should have no branches in
  the MST. This is evaluated by counting the number of vertices of
  degree two and comparing them to the total number of vertices
  (dropping those of degree one), \begin{equation}
  c_{\mathrm{stringy}} = \frac{|V^{(2)}|}{|V| - |V^{(1)}|}.
  \end{equation}
\item
  \textbf{Monotonic}: monotonicity is evaluated via the squared Spearman
  correlation coefficient, i.e.~the Pearson correlation between the
  ranks of \(x\) and \(y\), \begin{equation}
  c_{\mathrm{monotonic}} = r_{\mathrm{Spearman}}^2.
  \end{equation}
\end{itemize}

XXX maybe there is a problem with the implementation of outlying,
because this sometimes gives values larger than one? (e.g.~vs vs am on
mtcars, also gives NaN for several other measures: clumpy, striated,
stringy, monotonic)

\hypertarget{interface}{%
\subsection{Interface}\label{interface}}

The elementary function in the binostics package \texttt{scagnostics}
and can be called with a pair of vectors or a two-dimensional data
structure (a matrix or data frame).

The default S3 method is for a pair of vectors and will compute the
scagnostics measures for a single scatter plot. In this case additional
control and output. The additional arguments \texttt{bins} and
\texttt{outlierRmv} can be used for detailed checkes, but should not be
necessary for most applications. The output in this mode is a named list
\texttt{out}, with \texttt{out\$s} reporting the scagnostics measures
for the scatter plot, and \texttt{out\$bins} returns the binned data
used to compute them. Note that only non-empty bins are kept, and the
format is a \(3\times n\) matrix where the columns correspond to the bin
position in \(x, y\) and the bin count, and each row is a non-empty bin.
Internally the x and y axis is mapped to integer numbers, and the bin
size is reset if there are more than 250 non-empty bins, and this is
reflected in the matrix returned to the user.

In most applications we are interested in the scagnostics measures for
all combinations of variables in the input data. This is evaluated when
passing a matrix or data frame to the \texttt{scagnostics} function. In
this mode the function returns a data frame, where each row corresponds
to one combination of variables, and we have one column for each
scagnostics index. Two additional columns \texttt{var1} and
\texttt{var2} report the corresponding variable names.

Finally, we may only be interested in a single scagnostics index for a
pair of vectors. Since the measures all rely on the underlying binning
and triangulation we simply provide wrapper functions to access this
information conveniently. These functions are named according after the
corresponding scagnostics measure and report the index as an unnamed
scalar. This mode would be preferred e.g.~when using the measure as a
projection pursuit index.

\hypertarget{implementation-connecting-r-and-c}{%
\subsection{Implementation connecting R and
C++}\label{implementation-connecting-r-and-c}}

The interface is written in R and handels the reading the input, finds
all combinations of variables in the case of matrix input. For each pair
of variables it then pre-processes the data: entries with missing values
in either of the two variables are dropped, and each variable is
centered and scaled by its range. It is then using a direct call to C++
for the computation of the measures, as described below, and finally
collecting and formatting the output as described above.

The underlying C++ implementation allows for a fast evaluation, in
particular it handles the binning and triangulation. The triangulation
is used to determine the MST, convex hull and alpha hull, which are then
used to compute the measures. The calculation is done in the following
steps:

\begin{itemize}
\tightlist
\item
  Binning of the data: we use hexagonal binning, where the number of
  bins is a free parameter. Note that if this results in more than 250
  non-empty bins in the result, half the number of bins will be used
  instead.
\item
  Computing the Delaunay triangulation and the MST
\item
  Outlier detection: we use a cutoff on the edge length (see Eq.
  \ref{eq:w}), to identify outlying points. The triangulation and
  computation of the MST are repeated after removing the tagged
  outliers.
\item
  Compute scagnostics measures.
\end{itemize}

\hypertarget{example}{%
\subsection{Example}\label{example}}

A simple example showcasing how to use scagnostics measures

\begin{Schunk}
\begin{Sinput}
# FIXME need better examples!

library(tidyverse)
s <- binostics::scagnostics(mtcars)
s_tibble <- tibble::as_tibble(s[1:nrow(s),]) %>% #get tibble for plotting
  dplyr::mutate(vars = rownames(s))
GGally::ggpairs(select(s_tibble, -vars))
filter(s_tibble, Skinny==1 & Convex==1)$vars # these are discrete values only on the "outside"
filter(s_tibble, Skinny==1 & Convex==0)$vars # other discrete values give Convex=0
filter(s_tibble, Outlying==2 & Skewed==0)$vars # also points to discrete values only on the "outside"
\end{Sinput}
\end{Schunk}

\hypertarget{interactive-graphics}{%
\subsection{Interactive graphics}\label{interactive-graphics}}

Show how to use plotly to identify which variables lead to outliers on a
SPLOM of scagnostics measures?

\hypertarget{summary}{%
\subsection{Summary}\label{summary}}

\begin{itemize}
\tightlist
\item
  scagnostics measures useful when exploring large datasets
\item
  the binostics implementaton is efficient thanks to c++ interface, and
  portable (no java dependence)
\item
  most useful for interactive exploration (maybe good to use in Shiny
  app?)
\item
  connection with PP (cite PPI paper)
\end{itemize}

\bibliography{RJreferences}


\address{%
Ursula Laa\\
Affiliation\\
line 1\\ line 2\\
}
\href{mailto:author1@work}{\nolinkurl{author1@work}}

\address{%
Dianne Cook\\
Affiliation\\
line 1\\ line 2\\
}
\href{mailto:author2@work}{\nolinkurl{author2@work}}

\address{%
Hadley Wickham\\
Affiliation\\
line 1\\ line 2\\
}
\href{mailto:author3@work}{\nolinkurl{author3@work}}

\address{%
Heike Hofmann\\
Affiliation\\
line 1\\ line 2\\
}
\href{mailto:author4@work}{\nolinkurl{author4@work}}

